% Packages
\usepackage{hyperref}
\usepackage{siunitx}
\usepackage{bm} % Allows for proper bold math, such as in the slide titles
\usepackage{changepage} % Used to move tikzpicture to the left.
\usepackage[clock]{ifsym} % For work-in-progress symbol (clock)
\usepackage{tikz}
\usetikzlibrary{shapes,arrows}
\usetikzlibrary{fit,calc} % For creating block diagrams.
\usetikzlibrary{positioning}

% This is the title logo
% NOTE: This assumes the path to the file, but it is probably reasonable given
%       that it is most likely to be used in conjunction with pandoc
\pgfdeclareimage{beamerTitleLogoALICE}{$HOME/.local/share/pandoc/preamble/images/aliceLogo}
\pgfdeclareimage{beamerTitleLogoALICEForDarkBackground}{$HOME/.local/share/pandoc/preamble/images/2012-Jul-04-4_Color_Logo_small_DB_0.png}
\pgfdeclareimage{beamerTitleLogoLBL}{$HOME/.local/share/pandoc/preamble/images/4_BL_Horiz_Pos-rgb.png}
\pgfdeclareimage{beamerTitleLogoUCB}{$HOME/.local/share/pandoc/preamble/images/UCBerkeley_wordmark_blue}
\pgfdeclareimage{beamerTitleLogoJetscape}{$HOME/.local/share/pandoc/preamble/images/jetscapeLogo}
\pgfdeclareimage{beamerTitleLogoECCE}{$HOME/.local/share/pandoc/preamble/images/ecceLogo.png}
\pgfdeclareimage{beamerTitleLogoYale}{$HOME/.local/share/pandoc/preamble/images/yaleLogo}
\pgfdeclareimage{beamerTitleLogoWrightLab}{$HOME/.local/share/pandoc/preamble/images/wrightLab2Line}
% NOTE: The titlegraphic is defined in each document's header. This way, we can vary the logos more easily there.
%       We just need to enumerate the logos here.

% Customize beamer related settings
% Avoid showing a footer on the title slide.
% Needed for the metropolis theme, which by default doesn't have a footer, so it doesn't address this issue.
\setbeamertemplate{footline}{%
    \ifnum \insertframenumber=1%
      % There was no frame title
    \else%
      \begin{beamercolorbox}[wd=\textwidth, sep=2.5ex]{footline}%
        \usebeamerfont{page number in head/foot}%
        \usebeamertemplate*{frame footer}
        \hfill%
        %\usebeamertemplate*{frame numbering}
        \insertpagenumber
      \end{beamercolorbox}%
    \fi%
    }
% Show the outline at the beginning of each section
% See: https://tex.stackexchange.com/a/26713
\AtBeginSection[]
{
    \begin{frame}<beamer>
        \frametitle{Outline}
        \tableofcontents[currentsection]
    \end{frame}
}
% Customize the footer to include the short author followed by the date.
\setbeamertemplate{frame footer}{\insertshortauthor~-~\insertdate}
% Font customization. Somehow the font sizes look smaller in the green, so we increase a number of them.
\setbeamerfont{title}{size=\LARGE}
\setbeamerfont{frametitle}{size=\Large}
\setbeamerfont{author}{size=\normalsize}
% Set left and right margins
% See: https://tex.stackexchange.com/a/354025
\setbeamersize{text margin left=15mm,text margin right=15mm}

% Next, customize metropolis color scheme.
% Normally, we would just define normal text as UCBBlue and be done with it. However, the
% blue text is hard to read on slides. So we need to make the normal text black, and then update
% the theme were needed.
% Main colors
\definecolor{UCBBlue}{HTML}{003262} % UCB Blue (primary) - "Berkeley Blue"
\definecolor{UCBLightBlue}{HTML}{3B7EA1} % UCB Light Blue (secondary) - "Founder's Rock"
\definecolor{UCBOrange}{HTML}{D9661F} % UCB Orange (secondary) - "Wellman Tile"
\setbeamercolor{normal text}{fg=black, bg=white}
% Tweaks any bars, including dividing
\setbeamercolor{progress bar}{fg=UCBLightBlue, bg=UCBLightBlue!50!black!30} % Formula for bg is default in Metropolis.
% Title page
\setbeamercolor{titlelike}{fg=UCBBlue, bg=white}
\setbeamercolor{author}{fg=UCBBlue, bg=white}
\setbeamercolor{date}{fg=UCBBlue, bg=white}
\setbeamercolor{institute}{fg=UCBBlue, bg=white}
% Structure of the page (bullet point color, etc)
\setbeamercolor{structure}{fg=UCBBlue, bg=white}
% Color of the frame title and background, so we invert it (this appears to be the standard approach)
\setbeamercolor{palette primary}{fg=white, bg=UCBBlue}
% Footer color
\setbeamercolor{footline}{fg=UCBBlue!90}

% For drawing
% Generally, I use TikZ, but it's good to have reasonable settings when I use other things.
\setlength{\unitlength}{1in}
\thicklines

% Custom commands and helpers
% Images
\newcommand{\insertImage}[2]{\centering{\includegraphics[width=#1]{#2}}}

\newcommand\AliceApprovedLabel[4][1.0]{%
  % Arguments
  % 1: Scale factor used for including the image (prefactor for multiplying \textwidth). Default: 1.0.
  % 2: Name of node containing the image.
  % 3: Scale factor for the text size (it varies for Preliminary vs Simulation).
  % 4: Text to display.
  % Based on: https://tex.stackexchange.com/a/214390
  \begin{scope}[x={($ (#2.south east) - (#2.south west) $ )},y={( $ (#2.north west) - (#2.south west)$ )}, shift={(#2.south west)}]
      % Alice Label color: (178, 178, 178) (based on color picker)
      % pcr = Courier, which was identified as the font used by ALICE. (pcr in pdflatex, Courier in xelatex)
      % Placement determined by overlaying on the approved figure
      % Test with blue
      %\node[anchor=south west, inner sep={0.005 * #1 * \textwidth}] at (0, 0) {\fontfamily{pcr}\selectfont\resizebox{\textwidth*\real{#3}*\real{#1}}{!}{\textbf{#4}}};
      % Production with grey.
      \node[anchor=south west, inner sep={0.005 * #1 * \textwidth}] at (0, 0) {\setmonofont{Courier}\resizebox{\textwidth*\real{#3}*\real{#1}}{!}{\textcolor[rgb]{0.698,0.698,0.698}{\textbf{#4}}}};
      %\node[anchor=south west, inner sep={0.005 * #1 * \textwidth}] at (0, 0) {\setmonofont{Courier Bold}\resizebox{\textwidth*\real{#3}*\real{#1}}{!}{\textcolor[rgb]{0.698,0.698,0.698}{#4}}};
  \end{scope}
}
\newcommand\AlicePreliminary[3][1.0]{%
  \node[anchor=south west,inner sep=0] (image) at (0,0) {#2};
  \AliceApprovedLabel[#1]{image}{0.185}{#3}
}
\newcommand\AlicePreliminaryStandalone[3][1.0]{%
  \begin{tikzpicture}
    \AlicePreliminary[#1]{#2}{#3}
  \end{tikzpicture}
}

