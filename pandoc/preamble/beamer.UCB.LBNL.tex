% Packages
\usepackage{hyperref}
\usepackage{siunitx}
\usepackage{bm} % Allows for proper bold math, such as in the slide titles
\usepackage{changepage} % Used to move tikzpicture to the left.
\usepackage[clock]{ifsym} % For work-in-progress symbol (clock)

% This is the title logo
% NOTE: This assumes the path to the file, but it is probably reasonable given
%       that it is most likely to be used in conjunction with pandoc
\pgfdeclareimage{beamerTitleLogoALICE}{$HOME/.local/share/pandoc/preamble/images/aliceLogo}
\pgfdeclareimage{beamerTitleLogoALICEForDarkBackground}{$HOME/.local/share/pandoc/preamble/images/2012-Jul-04-4_Color_Logo_small_DB_0.png}
\pgfdeclareimage{beamerTitleLogoLBL}{$HOME/.local/share/pandoc/preamble/images/4_BL_Horiz_Pos-rgb.png}
\pgfdeclareimage{beamerTitleLogoUCB}{$HOME/.local/share/pandoc/preamble/images/UCBerkeley_wordmark_blue}
\pgfdeclareimage{beamerTitleLogoJetscape}{$HOME/.local/share/pandoc/preamble/images/jetscapeLogo}
\pgfdeclareimage{beamerTitleLogoECCE}{$HOME/.local/share/pandoc/preamble/images/ecceLogo.png}
\pgfdeclareimage{beamerTitleLogoYale}{$HOME/.local/share/pandoc/preamble/images/yaleLogo}
\pgfdeclareimage{beamerTitleLogoWrightLab}{$HOME/.local/share/pandoc/preamble/images/wrightLab2Line}
% NOTE: The titlegraphic is defined in each document's header. This way, we can vary the logos more easily there.
%       We just need to enumerate the logos here.

% Customize beamer related settings
% Avoid showing a footer on the title slide.
% Needed for the metropolis theme, which by default doesn't have a footer, so it doesn't address this issue.
\setbeamertemplate{footline}{%
    \ifnum \insertframenumber=1%
      % There was no frame title
    \else%
      \begin{beamercolorbox}[wd=\textwidth, sep=2.5ex]{footline}%
        \usebeamerfont{page number in head/foot}%
        \usebeamertemplate*{frame footer}
        \hfill%
        %\usebeamertemplate*{frame numbering}
        \insertpagenumber
      \end{beamercolorbox}%
    \fi%
    }
% Show the outline at the beginning of each section
% See: https://tex.stackexchange.com/a/26713
\AtBeginSection[]
{
    \begin{frame}<beamer>
        \frametitle{Outline}
        \tableofcontents[currentsection]
    \end{frame}
}
% Customize the footer to include the short author followed by the date.
\setbeamertemplate{frame footer}{\insertshortauthor~-~\insertdate}
% Font customization. Somehow the font sizes look smaller in the green, so we increase a number of them.
\setbeamerfont{title}{size=\LARGE}
\setbeamerfont{frametitle}{size=\Large}
\setbeamerfont{author}{size=\normalsize}
% Set left and right margins
% See: https://tex.stackexchange.com/a/354025
\setbeamersize{text margin left=15mm,text margin right=15mm}

% Next, customize metropolis color scheme.
% Normally, we would just define normal text as UCBBlue and be done with it. However, the
% blue text is hard to read on slides. So we need to make the normal text black, and then update
% the theme were needed.
% Main colors
\definecolor{UCBBlue}{HTML}{003262} % UCB Blue (primary) - "Berkeley Blue"
\definecolor{UCBLightBlue}{HTML}{3B7EA1} % UCB Light Blue (secondary) - "Founder's Rock"
\definecolor{UCBOrange}{HTML}{D9661F} % UCB Orange (secondary) - "Wellman Tile"
\setbeamercolor{normal text}{fg=black, bg=white}
% Tweaks any bars, including dividing
\setbeamercolor{progress bar}{fg=UCBLightBlue, bg=UCBLightBlue!50!black!30} % Formula for bg is default in Metropolis.
% Title page
\setbeamercolor{titlelike}{fg=UCBBlue, bg=white}
\setbeamercolor{author}{fg=UCBBlue, bg=white}
\setbeamercolor{date}{fg=UCBBlue, bg=white}
\setbeamercolor{institute}{fg=UCBBlue, bg=white}
% Structure of the page (bullet point color, etc)
\setbeamercolor{structure}{fg=UCBBlue, bg=white}
% Color of the frame title and background, so we invert it (this appears to be the standard approach)
\setbeamercolor{palette primary}{fg=white, bg=UCBBlue}
% Footer color
\setbeamercolor{footline}{fg=UCBBlue!90}

% For drawing
% Generally, I use TikZ, but it's good to have reasonable settings when I use other things.
%\setlength{\unitlength}{1in}
%\thicklines

